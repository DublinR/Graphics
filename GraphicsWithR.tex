\documentclass[11pt]{article} % use larger type; default would be 10pt

\usepackage[utf8]{inputenc} % set input encoding (not needed with XeLaTeX)
\usepackage{geometry} % to change the page dimensions
\geometry{a4paper} % or letterpaper (US) or a5paper or....
\usepackage{graphicx} % support the \includegraphics command and options

% \usepackage[parfill]{parskip} % Activate to begin paragraphs with an empty line rather than an indent
\usepackage{framed}
%%% PACKAGES
\usepackage{booktabs} % for much better looking tables
\usepackage{array} % for better arrays (eg matrices) in maths
\usepackage{paralist} % very flexible & customisable lists (eg. enumerate/itemize, etc.)
\usepackage{verbatim} % adds environment for commenting out blocks of text & for better verbatim
\usepackage{subfig} % make it possible to include more than one captioned figure/table in a single float
% These packages are all incorporated in the memoir class to one degree or another...

%%% HEADERS & FOOTERS
\usepackage{fancyhdr} % This should be set AFTER setting up the page geometry
\pagestyle{fancy} % options: empty , plain , fancy
\renewcommand{\headrulewidth}{0pt} % customise the layout...
\lhead{}\chead{}\rhead{}
\lfoot{}\cfoot{\thepage}\rfoot{}
\usepackage{sectsty}
\allsectionsfont{\sffamily\mdseries\upshape} % (See the fntguide.pdf for font help)
% (This matches ConTeXt defaults)

%%% ToC (table of contents) APPEARANCE
\usepackage[nottoc,notlof,notlot]{tocbibind} % Put the bibliography in the ToC
\usepackage[titles,subfigure]{tocloft} % Alter the style of the Table of Contents
\renewcommand{\cftsecfont}{\rmfamily\mdseries\upshape}
\renewcommand{\cftsecpagefont}{\rmfamily\mdseries\upshape} % No bold!

%%% END Article customizations

%%% The "real" document content comes below...

\title{Graphics with \texttt{R}}
\author{Kevin O'Brien}


\begin{document}

\section{Overview}
This material does \textbf{not} cover \textbf{\textit{ggplot2}}.
\section{Colours supported by \texttt{R}}
\begin{framed}
\begin{verbatim}
colours()
colors()
\end{verbatim}
\end{framed}
A list of the 657 colours supported by \texttt{R}.
\begin{verbatim}
  [1] "white"                "aliceblue"            "antiquewhite"        
  [4] "antiquewhite1"        "antiquewhite2"        "antiquewhite3"       
  [7] "antiquewhite4"        "aquamarine"           "aquamarine1"    
....
[358] "grey97"               "grey98"               "grey99"              
[361] "grey100"              "honeydew"             "honeydew1"           
[364] "honeydew2"            "honeydew3"            "honeydew4"           
[367] "hotpink"              "hotpink1"             "hotpink2"            
[370] "hotpink3"             "hotpink4"             "indianred"           
[373] "indianred1"           "indianred2"           "indianred3"                   
....
[649] "wheat3"               "wheat4"               "whitesmoke"          
[652] "yellow"               "yellow1"              "yellow2"             
[655] "yellow3"              "yellow4"              "yellowgreen"     
\end{verbatim}

\begin{verbatim}
> colours()[1:10]
 [1] "white"         "aliceblue"     "antiquewhite"  "antiquewhite1"
 [5] "antiquewhite2" "antiquewhite3" "antiquewhite4" "aquamarine"   
 [9] "aquamarine1"   "aquamarine2"
\end{verbatim}
\newpage
\section{Plotting with \texttt{R}}

%---------------------------------------------------------------------------------------------%
\subsection{Scatterplots}


%---------------------------------------------------------------------------------------------%
\subsection{Randomly Coloured Grids}

\end{document}
