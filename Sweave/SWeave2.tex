\documentclass[11pt]{article}
\usepackage{graphicx, verbatim}
\setlength{\textwidth}{6.5in} 
\setlength{\textheight}{9in}
\setlength{\oddsidemargin}{0in} 
\setlength{\evensidemargin}{0in}
\setlength{\topmargin}{-1.5cm}

\begin{document}
\SweaveOpts{concordance=TRUE}
\begin{center}
{\bf \Large Stat 500 Assignment 1\\}
 \end{center}
 
 Using the rock data in the datasets package. First we do some plots:\\
 %%\setkeys{Gin}{width=.3\linewidth}
<<>>=
 data(rock)
  require(ggplot2)
  plot1 = qplot(x=area, y=log(perm), data=rock) + theme_bw() + geom_smooth(col="red")
   print(plot1 )
@
%%  lattice and ggplots must be inside a print statement to show up
<<>>=
   print(plot1 + aes(x = peri) )
@
<<>>=
   print(plot1 + aes(x = shape) )
@

\end{document}

 Now go back and remove the \%\% comments on the line 15 above here  
 to set each plot width to .3
 times linewidth, and the three plots should fit on one line.  If you leave a blank line between the three code chunks above, they will start on new lines.

Summary of the linear model:
<<>>=
  rock.lmfit <- lm(log(perm) ~ ., rock)
  summary(rock.lmfit)
@

<<>>=
 xtable::xtable( summary(rock.lmfit)$coef, digits=5)
@
<<>>=
 rock.rlmfit = MASS::rlm( log(perm) ~ ., rock)
 xtable::xtable( summary(rock.rlmfit)$coef, digits = 4)
  ## assumes that you have the xtable package available. It creates latex tables.
@

%To print any R object use a \verb|\Sexpr{any_R_object}| statement like this:
%AIC of the linear model is \Sexpr{AIC(rock.lmfit)}.  
 
 
